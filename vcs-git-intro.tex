\documentclass{beamer}
\usetheme{Boadilla}

\title{Version control and Git}
\author{Augusto Fraga Giachero}
\date{\today}

\AtBeginSection[]{
  \begin{frame}
  \vfill
  \centering
  \begin{beamercolorbox}[sep=8pt,center,shadow=true,rounded=true]{title}
    \usebeamerfont{title}\insertsectionhead\par
  \end{beamercolorbox}
  \vfill
  \end{frame}
}

\begin{document}

% Title page frame
\begin{frame}
  \titlepage 
\end{frame}

% Outline frame
\begin{frame}{Outline}
  \tableofcontents
\end{frame}

\section{Version Control}

\subsection{In the Beginning}
\begin{frame}{In the Beginning}
  \begin{itemize}
    \item Collaborative software development required fine coordination between developers;
    \item Very hard to scale to large development teams;
    \item Versioning was a manual operation and error prone;
    \item No easy way to pinpoint modifications that introduced bugs;
    \item To avoid accidental data loss, backups had to be made often.
  \end{itemize}
\end{frame}

\subsection{Enter Version Control}
\begin{frame}{Enter Version Control}
  \begin{itemize}
    \item First widely used VCS: SCCS (developed in 1972 for OS/360, ported to UNIX afterwards, could only track single files);
    \item RCS, CVS and SVN followed (could track multiple files);
    \item Distributed VCS like BitKeeper, Git, Mercurial and Bazaar entered the scene a few years later.
  \end{itemize}
\end{frame}

\subsection{Version Control Concepts}
\begin{frame}{Version Control Concepts}
  \begin{itemize}
    \item Software to track modifications of a single file or a set of files (repository);
    \item Presents a history of states (commits) of a repository;
    \item Can store different history lines for the same repository (branches);
    \item Focused on tracking of plain-text files, but can offer binary file support, albeit with some limitations.
  \end{itemize}
\end{frame}

\subsection{Centralized Version Control}
\begin{frame}{Centralized Version Control}
  \begin{itemize}
    \item Requires a central server for almost all version control operations;
    \item Generally have the capability to 'lock files' to avoid concurrent modifications to the same file;
    \item Network access is required for collaborative development;
    \item No local branches.
  \end{itemize}
\end{frame}

\subsection{Distributed Version Control}
\begin{frame}{Distributed Version Control}
  \begin{itemize}
    \item There is no central server that coordinates commits, merges and other version control operations;
    \item Everything can be done locally, no network access required;
    \item All developers have a full copy of the repository history, resulting in a 'distributed backup' of the repository.
  \end{itemize}
\end{frame}

\section{Git}

\subsection{Git Design Goals}
\begin{frame}{Git Design Goals}
  \begin{itemize}
    \item High performance; 
    \item Data integrity;
    \item Support distributed, non-linear workflows;
    \item Easy branching.
  \end{itemize}
\end{frame}

\end{document}
